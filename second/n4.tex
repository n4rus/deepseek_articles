\documentclass{article}
\usepackage{amsmath, amssymb}
\usepackage{graphicx}
\usepackage{url}
\usepackage{natbib}
\usepackage[margin=1in]{geometry}

\title{Dark Matter, Dark Energy, and the Electromagnetic Past: A Framework for Cosmic Origins}
\author{Author Name \\ Collaboration with ChatGPT (OpenAI)}
\date{\today}

\begin{document}

\maketitle

\begin{abstract}
This paper reinterprets dark matter and dark energy as residual electromagnetic energy from the past, proposing that the observable universe is a time-delayed projection of entangled quantum states. The framework posits that the Big Bang originated from a singularity in a quantum void, with inflation driven by a white hole-like reversal of spacetime curvature. Forces are derived from radiative interactions between particles and waves across time frames, and the initial singularity is modeled as a self-entangling virtual particle in flat spacetime. Comparisons to quantum field theory, general relativity, and cosmic microwave background (CMB) observations are provided.
\end{abstract}

\section{Introduction}
Traditional models treat dark matter and dark energy as distinct entities, but this paper reframes them as electromagnetic energy from the past, projected onto our observational "light cone." The universe's initial singularity is hypothesized to arise from a quantum fluctuation in a void, with inflation modeled as a white hole reversing black hole dynamics. Forces are calculated as radiative interactions across delayed time frames, requiring corrections to observed particle positions based on their past energy states.

\section{Key Framework}
\subsection{Dark Matter and Dark Energy as Electromagnetic Remnants}
Dark matter and dark energy are reinterpreted as time-delayed electromagnetic radiation from earlier cosmic epochs. The observed CMB temperature \( T_{\text{CMB}} = 2.725 \, \text{K} \) is a relic of this energy, with dark matter arising from entangled photon pairs decohered over time:
\[
\rho_{\text{DM}} = \int_{t_{\text{BB}}}^{t_0} \epsilon_{\gamma}(t) \, e^{-\lambda (t_0 - t)} dt,
\]
where \( \epsilon_{\gamma}(t) \) is the photon energy density at time \( t \), and \( \lambda \) quantifies decoherence.

\subsection{Force Equation in a Quantum Void}
The force equation balances radiative interactions across time-delayed positions. For a universe beginning in a void, the net force at the singularity is zero:
\[
F = \sum_i \left( \frac{q_i E_i(t - \Delta t_i)}{c^2} + \frac{G m_i m_j}{r_{ij}^2(t - \Delta t_i)} \right) - \Lambda(t) = 0,
\]
where \( \Delta t_i = \frac{r_i}{c} \) is the light-travel delay, and \( \Lambda(t) \) represents dark energy as residual inflationary pressure.

\subsection{Singularity Formation and White Hole Inflation}
The initial singularity forms when a virtual particle pair in a quantum void entangles and collapses:
\[
\Delta x \Delta p \sim \hbar \quad \Rightarrow \quad \rho_{\text{virtual}} \geq \rho_{\text{Planck}}.
\]
Inflation is modeled as a white hole reversing black hole thermodynamics:
\[
ds^2 = -e^{2\alpha t} dt^2 + e^{2\beta t} d\vec{x}^2,
\]
where \( \alpha > 0 \) (expansion) replaces \( \alpha < 0 \) (collapse) in black hole metrics. Matter and energy "spill out" from the white hole horizon.

\section{Comparison to Established Theories}
\subsection{Quantum Field Theory (QFT)}
\begin{itemize}
    \item \textbf{Virtual Particles}: The singularity's origin aligns with QFT vacuum fluctuations but requires entanglement to sustain collapse.
    \item \textbf{Conflict}: QFT predicts virtual particle annihilation, not sustained collapse. This requires extending QFT with nonlocal entanglement.
\end{itemize}

\subsection{General Relativity (GR)}
\begin{itemize}
    \item \textbf{White Hole Inflation}: Mathematically consistent with GR's time-reversible equations but contradicts the second law of thermodynamics (white holes violate entropy increase).
    \item \textbf{Resolution}: Assume inflation resets entropy, akin to the Conformal Cyclic Cosmology.
\end{itemize}

\subsection{Cosmic Microwave Background (CMB)}
\begin{itemize}
    \item \textbf{Support}: CMB isotropy aligns with past electromagnetic energy homogenized over time.
    \item \textbf{Conflict}: CMB acoustic peaks require cold dark matter, not delayed photons. Proposed fix: Decohered photons acquire effective mass \( m_{\gamma} \sim \hbar \lambda / c^2 \).
\end{itemize}

\subsection{Entanglement and Holography}
\begin{itemize}
    \item \textbf{AdS/CFT}: The framework resembles holographic duality, where boundary fields (past electromagnetic states) project onto bulk spacetime.
    \item \textbf{Test}: Compare entanglement entropy of CMB polarization maps to AdS/CFT predictions.
\end{itemize}

\section{Experimental Tests}
\subsection{Time-Delayed Force Corrections}
Measure gravitational lensing discrepancies when accounting for source positions at emission time \( t - \Delta t \):
\[
\delta \theta = \theta_{\text{obs}} - \theta_{\text{emitted}}.
\]
Prediction: \( \delta \theta \propto \Delta t \, H_0 \).

\subsection{White Hole Signatures}
Search for imprints of white hole thermodynamics in the CMB:
\begin{itemize}
    \item Circular polarization anomalies from reversed electromagnetic modes.
    \item Vortex structures in large-scale galaxy surveys.
\end{itemize}

\subsection{Decohered Photon Mass}
Detect \( m_{\gamma} \) via frequency-dependent photon arrival delays in gamma-ray bursts:
\[
\Delta t \propto m_{\gamma}^2 \frac{D}{E^2},
\]
where \( D \) is the distance and \( E \) the photon energy.

\section{Critical Analysis}
\subsection{Strengths}
\begin{itemize}
    \item Unifies dark matter, dark energy, and CMB under one electromagnetic framework.
    \item Mathematically consistent with GR and QFT if nonlocal entanglement is permitted.
\end{itemize}

\subsection{Weaknesses}
\begin{itemize}
    \item \textbf{Entanglement Sustainability}: No mechanism prevents virtual particle annihilation.
    \item \textbf{White Hole Thermodynamics}: Requires ad hoc entropy reset.
    \item \textbf{Photon Mass}: Contradicts precision tests (\( m_{\gamma} < 10^{-27} \, \text{eV} \)).
\end{itemize}

\section{Conclusion}
The framework provides a provocative reinterpretation of dark matter, dark energy, and cosmic origins but requires extensions to QFT and GR to resolve thermodynamic and observational conflicts. Experimental tests of time-delayed forces, white hole imprints, and photon mass could validate or refute the model.

\section*{Acknowledgments}
The author acknowledges ChatGPT (OpenAI) for refining equations, proposing tests, and identifying thermodynamic inconsistencies.

\bibliographystyle{plainnat}
\bibliography{references}

\end{document}
