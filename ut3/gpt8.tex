%%%%%%%%%%%%%%%%%%%%%%%%%%%%%%%%%%%%%%%%%%%%%%%%%%%%%%%%%%%%%%%%%%%%%%%%%%%%%%
%                              PREAMBLE
%%%%%%%%%%%%%%%%%%%%%%%%%%%%%%%%%%%%%%%%%%%%%%%%%%%%%%%%%%%%%%%%%%%%%%%%%%%%%%
\documentclass[12pt, a4paper]{article}

% Standard packages (duplicates removed)
\usepackage{amsmath, amssymb, amsthm}    % Advanced math and theorem environments
\usepackage{graphicx}                   % For figures
\usepackage{url}                        % For URLs
\usepackage[margin=1in]{geometry}       % For page margins
\usepackage{float}                      % For figure/table positioning
\usepackage{siunitx}                    % For units and numbers
                   % For bibliography management
\usepackage{tikz}                       % For creating diagrams
\usepackage{physics}
\usepackage{biblatex}
\begin{document}
\addbibresource{references.bib} % Reference file

\title{Comprehensive Study on High-Energy Gamma-Ray Bursts: Theory, Observations, and Revised Models}
\author{Your Name}
\date{\today}

\begin{document}

\maketitle

\begin{abstract}
This paper presents a thorough analysis of high-energy gamma-ray bursts (GRBs), merging theoretical derivations, observational data, and model refinements. By analyzing spectral deviations, we refine the theoretical framework to align with the observed burst GRB 190114C, incorporating relativistic effects, synchrotron emission, and inverse Compton scattering. We derive fundamental equations, validate models with experimental data, and propose a revised energy distribution that improves agreement with observations.
\end{abstract}

\section{Introduction}
Gamma-ray bursts (GRBs) are the most energetic events in the universe, originating from relativistic jets powered by stellar collapses or neutron star mergers. Observations indicate that these bursts exhibit power-law spectra, consistent with non-thermal emission mechanisms. In this work, we analyze theoretical predictions against observed data and refine the model for better accuracy.

\section{Mathematical Framework and Model Derivation}
The energy distribution of accelerated electrons in GRB jets follows a power-law:
\begin{equation}
N(E_e) \propto E_e^{-p},
\end{equation}
which leads to a photon flux:
\begin{equation}
F(E) \propto E^{-(p+1)/2}.
\end{equation}
Observations of GRB 190114C indicate a spectral index of $-2.2$, implying $p = 3.4$.

To incorporate high-energy cutoffs, we modify the flux equation:
\begin{equation}
F(E) = C E^{-2.2} e^{-E/E_{\text{cut}}},
\end{equation}
where $E_{\text{cut}}$ accounts for synchrotron losses.

\section{Comparison with Observations}
GRB 190114C was observed at TeV energies, showing a deviation from previous theoretical models. Our revised function better aligns with the experimental data (see Figure \ref{fig:comparison}).

\begin{figure}[h]
    \centering
    \includegraphics[width=0.7\textwidth]{gamma_burst_comparison.png}
    \caption{Comparison of observed and revised theoretical spectra for GRB 190114C.}
    \label{fig:comparison}
\end{figure}

\section{Discussion and Conclusion}
This work highlights the necessity of refining theoretical models to include additional energy losses and relativistic corrections. Our revised model, incorporating a power-law index of $p=3.4$ and an exponential cutoff, significantly improves agreement with observations. Future studies should further explore GRB emission mechanisms through multi-wavelength observations.

\printbibliography

\end{document}

